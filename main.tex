\documentclass{article}
\usepackage[utf8]{inputenc}
\usepackage{hyperref}
\usepackage{listings}

\title{SR Peru Development Setup Instructions}
\author{ Jared Olson and Brett Fox }
\date{April 2018}

\begin{document}

\maketitle

\newpage
\tableofcontents
\newpage

\section{Install Necessary Packages}
\subsection{Mac}
\textbf{Install Homebrew:}
\begin{verbatim}
    $ /usr/bin/ruby -e "$(curl -fsSL 
    https://raw.githubusercontent.com/Homebrew/install/master/install)"
\end{verbatim}
\textbf{Update Homebrew:}
\begin{verbatim}
    $ brew update
\end{verbatim}
\textbf{Install Packages:}
\begin{verbatim}
    $ brew install python2.7
    $ brew install postgresql
    $ brew install postgis
\end{verbatim}

\subsection{Ubuntu}
\textbf{Install Packages:}
\begin{verbatim}
    $ apt-get install python2.7
    $ apt-get install postgresql
    $ apt-get install postgis
    $ apt-get install python-pip
\end{verbatim}

\section{Virtual Environment}
\textbf{Download and set up virtualenv and virtualenvwrapper}
\begin{verbatim}
    $ pip install virtualenv
    $ pip install virtualenvwrapper
\end{verbatim}
Make a directory to store your virtual environments:
\begin{verbatim}
    $ mkdir /opt/venv
\end{verbatim}
Edit your .bashrc file by opening it using:
\begin{verbatim}
    $ vi ~/.bashrc
\end{verbatim}
Add the following lines to the file (don't forget to save before exiting):
\begin{verbatim}
    # Set the virtualenv directory
    export WORKON_HOME=/opt/venv
    
    # Tell VirtualEnvWrapper where Python is located
    VIRTUALENVWRAPPER_PYTHON='/usr/bin/python'
    
    # Load the virtualenvwrapper script
    source /usr/local/bin/virtualenvwrapper.sh
\end{verbatim}
Source your .bashrc file:
\begin{verbatim}
    $ source ~/.bashrc
\end{verbatim}
\textbf{Make virtualenv:}
\\Make sure you are using python2.7 in your virtual environment.
\begin{verbatim}
    $ mkvirtualenv sr_peru
\end{verbatim}
\textbf{Set your Django settings module in your virtualenv:}
\\Open the postactivate file using (you may need to be root):
\begin{verbatim}
    $ vi ~/your_path/sr_peru/bin/postactivate/
\end{verbatim}
Add the following to the open file (don't forget to save before exiting):
\begin{verbatim}
    export DJANGO_SETTINGS_MODULE='sr_peru.settings.development'
\end{verbatim}
Open the postdeactivate file using (you may need to be root):
\begin{verbatim}
    $ vi ~/your_path/sr_peru/bin/postdeactivate/
\end{verbatim}
Add the following to the open file (don't forget to save before exiting):
\begin{verbatim}
    unset DJANGO_SETTINGS_MODULE
\end{verbatim}
Deactivate the virtualenv for changes to take affect and activate again:
\begin{verbatim}
    $ deactivate
    $ workon sr_peru
\end{verbatim}

\section{Setup The Project}
Navigate to your where you want your project to reside and clone the repository:
\begin{verbatim}
    $ git clone https://github.com/crcresearch/sr_peru.git
\end{verbatim}
Install PIP packages:
\begin{verbatim}
    $ pip install -r sr_peru/sr_peru/requirements/development.txt
\end{verbatim}
\section{Create A New Database}
Open psql:
\begin{verbatim}
    $ sudo su
    $ su postgres
    $ psql postgres
\end{verbatim}
Run the following commands while in psql:
\begin{verbatim}
    # CREATE ROLE django WITH CREATEDB LOGIN;
    # CREATE DATABASE iquitos WITH owner django;
    # \c iquitos
    # CREATE EXTENSION postgis;
    # \q
\end{verbatim}
At this point you will need a dump of the GIS schema.  For security reasons, one is not provided so if you do not have access to the database to get a dump yourself then contact the team at UCDavis. When you have the database dump navigate to the folder that contains it and run (you will get a lot of errors about roles not existing and that is okay):
\begin{verbatim}
    $ psql -d iquitos -f gis.sql
\end{verbatim}
Connect to the iquitos database:
\begin{verbatim}
    $ psql iquitos
\end{verbatim}
Once connected run the following commands:
\begin{verbatim}
    # GRANT USAGE ON SCHEMA gis TO django;
    # GRANT ALL PRIVILEGES ON ALL TABLES IN SCHEMA gis TO django;
    # ALTER ROLE django SET search_path TO public, gis;
    # \q
\end{verbatim}
\section{Create A local.py Settings File}
Get a secret key that we will put into our local settings file. Open Python and run the following command. Copy the output.
\begin{verbatim}
    >>> import random
    >>> ''.join(random.SystemRandom().choice('abcdefghijklmnopqrstuvwxyz0123
    456789!@#$%^&*(-_=+)') for i in range(50))
\end{verbatim}
Now create a local settings file using:
\begin{verbatim}
    $ vi /your_path/sr_peru/sr_peru/settings/local.py
\end{verbatim}
Add the following to the new file:
\begin{verbatim}
    # SECURITY WARNING: keep the secret key used in production secret!
    SECRET_KEY = 'secret_key'
    
    DATABASES = {
        'default': {
            'ENGINE': 'django.db.backends.postgresql_psycopg2',
            'NAME': 'iquitos',
            'USER': 'django',
            'PASSWORD': 'password',
            'HOST': '127.0.0.1',
            'PORT': '5432',
        }
    }

    DEFAULT_PASSWORD = 'kikiriki'
\end{verbatim}
Navigate to the project root directory and run the following commands:
\begin{verbatim}
    $ python manage.py migrate
    $ python manage.py load_fixtures
    $ python manage.py compilemessages
\end{verbatim}
\end{document}
